\documentclass{sig-alternate-br}
\usepackage{multirow}
\usepackage{rotating}
\usepackage{xcolor,colortbl}
\usepackage[official]{eurosym}
\usepackage{hyperref}

\begin{document}

\CopyrightYear{2013} 

\title{Determining the optimal locking strategy for your database}

\numberofauthors{2}
\author{
\alignauthor Niek Tax\\
       \affaddr{University of Twente}\\
       \affaddr{P.O. Box 217, 7500AE Enschede}\\
       \affaddr{The Netherlands}\\
\alignauthor Ruud Verbij\\
       \affaddr{University of Twente}\\
       \affaddr{P.O. Box 217, 7500AE Enschede}\\
       \affaddr{The Netherlands}\\
}

\maketitle
\begin{abstract}
\end{abstract}

\keywords{Locking strategies}

% INTRODUCTION
\section{Introduction}
\label{sec:intro}
This paper shows the impact of locking strategies on the performance of a database system. Databases allow for two types of locking strategies, namely optimistic and pessimistic. Each of these two strategies has advantages and disadvantages, depending on the number of conflicting updates that could occur. By introducing a metric which describes this level of conflicting updates in a set of transactions, one can easily define the type of transactions one has. This paper will present the break-even point for this metric, describing when to switch from a optimistic to a pessimistic locking strategy within the database. The results from this paper can be used when deciding upon the optimal strategy.

This paper is structured as follows, first the case that was used for this paper will be described in Section~\ref{sec:case}, the methodology will be described in Section~\ref{sec:methodology}. The results presented in Section~\ref{sec:results}. Finally, this paper will end with a conclusion and discussion in Section~\ref{sec:conclusions}.

% CASE
\section{Case description}
\label{sec:case}
The show case for this paper will be a small database for a bank consisting of $10.000$ account holders for $13.000$ accounts. Accounts carry out financial transactions mutual exchanges of money with other accounts (of potentially other account holders). This show case models the number of transactions an account makes via the normal distribution: some accounts hardly ever send money to other accounts, while some accounts have a really frequent exchange of money, but most of the accounts will exchange money about average. The receiving end of the transactions is modeled similarly using the normal distribution. The link between sender and receiver is randomly distributed.

% METHODOLOGY
\section{Methodology}
\label{sec:methodology}
Transactions of a bank database will be simulated using a simple model of a bank database consisting of \emph{bank account} and \emph{account holder} tables in a PostgreSQL database. Multiple sets of financial transactions will be generated such that the sender and the receiver side of the transactions will hold a $\mathcal{N} (m,\sigma^2)$ distribution, varying on the \emph{m} parameter. Higher values of \emph{m} result in higher expected values of read-write conflicts, because of higher transaction-denseness. 

The metric used in this paper will be that \emph{aborted transaction percentage}, which indicates the percentage of all transactions that are aborted while using the optimistic locking strategy.

The generated financial transaction sets will be executed on the database once with an optimistic locking strategy and once with a pessimistic locking strategy, where each transaction will be randomly assigned to one of 20 threads. The hypothesis is the optimistic locking strategies to be more time-efficient for less transaction-dense sets, where pessimistic locking strategies are expected to be more time-efficient for transaction-dense sets. If this hypothesis is correct there will be a break-even point between optimistic and pessimistic locking strategy performance for some aborted transactions percentage. By measuring execution times of the transaction sets, varied on \emph{m}, one can identify this break-even point and thereby find a solution to the problem of finding the best performing locking strategy for a given level of transaction density.

\section{Results}
\label{sec:results}

% CONCLUSIONS AND DISCUSSION
\section{Conclusions and Discussion}
\label{sec:conclusions}

\bibliographystyle{plainnat}
%\bibliographystyle{IEEEtran}
\bibliography{IEEEabrv,literature}

\balancecolumns

\onecolumn

\end{document}
